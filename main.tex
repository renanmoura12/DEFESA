\documentclass[12pt, a4paper]{report}

%----------------- MARGENS -----------------
\usepackage[top=3cm, bottom=2cm, right=2cm, left=3cm]{geometry}

%----------------- ENCODING E LÍNGUA -----------------
\usepackage[utf8]{inputenc}
\usepackage[T1]{fontenc}
\usepackage[brazil]{babel}
\usepackage{csquotes}

%----------------- FIGURAS E GRÁFICOS -----------------
\usepackage{graphicx}
\usepackage{float}
\usepackage[labelsep=endash]{caption}
\usepackage{needspace}
\usepackage{placeins}
\usepackage{tikz}
\usetikzlibrary{arrows.meta,bending}
\usepackage[all]{xy}
\usepackage{capt-of}
\usepackage{pdfpages}

% Ambiente quadro
\usepackage{newfloat}
\floatstyle{plain}
\newfloat{quadro}{htbp}{loq}
\floatname{quadro}{Quadro}
\captionsetup[quadro]{position=top}
\renewcommand{\thequadro}{\arabic{quadro}}

%----------------- TEXTO E ESPAÇAMENTO -----------------
\usepackage{indentfirst}
\usepackage{setspace}
\linespread{1.5}
\usepackage{enumitem}
\usepackage{blindtext}
\usepackage[titletoc]{appendix}
\usepackage{hyphenat}
\usepackage{tabularx}
\usepackage{array}
\usepackage{ragged2e}

%----------------- MATEMÁTICA -----------------s
\usepackage{amsmath, amssymb, amsfonts, amsthm}
\usepackage{physics}
\usepackage{MnSymbol}
\usepackage{thmtools}
\usepackage{newtxtext,newtxmath}
\usepackage{cancel}
\usepackage{chngcntr}
\counterwithout{figure}{chapter}
\counterwithout{table}{chapter}


\DeclareMathOperator{\mdc}{mdc}
\DeclareMathOperator{\sen}{sen}

%----------------- NOMENCLATURA -----------------
\usepackage{nomencl}
\makenomenclature
\setlength{\nomlabelwidth}{3cm}
\setlength{\nomitemsep}{-\parsep}
\renewcommand{\nomname}{\normalsize \makebox[\linewidth]{\MakeUppercase{Lista de símbolos}}}

%----------------- HYPERLINKS -----------------
\usepackage[pdftex]{hyperref}

%----------------- CITAÇÕES ABNT -----------------
\usepackage[backend=biber,style=abnt,sorting=nyt]{biblatex}
\addbibresource{referencias.bib}

\usepackage{threeparttable}

%----------------- TÍTULOS -----------------
\usepackage{titlesec}
\titleformat*{\section}{\normalsize \bfseries}
\titleformat*{\subsection}{\normalsize}
\titleformat*{\subsubsection}{\normalsize \bfseries}
\titleformat{\chapter}{\bfseries}{\filright \thechapter}{20pt}{\filright}

\titlespacing*{\chapter}{0pt}{0pt}{*1.5}
\titlespacing{\section}{0pt}{*1.5}{*1.5}

%----------------- SUMÁRIO -----------------
\usepackage{tocloft}
\renewcommand\cftchapnumwidth{3.8em}
\renewcommand\cftsecnumwidth{3.8em}
\renewcommand\cftsubsecnumwidth{3.8em}
\renewcommand{\cftsecfont}{\bfseries}
\setlength{\cftsecindent}{0pt}
\setlength{\cftsubsecindent}{0pt}
\renewcommand{\cfttoctitlefont}{\hfil \normalfont \centering \bfseries\MakeUppercase}
\renewcommand{\cftaftertoctitleskip}{0pt}

%----------------- TEOREMAS -----------------
\newtheorem{theorem}{Teorema}[chapter]
\newtheorem{lemma}[theorem]{Lema}
\newtheorem{corollary}{Corolário}[theorem]

\theoremstyle{definition}
\newtheorem{definition}{Definição}[chapter]
\newtheorem{exemplo}{Exemplo}[chapter]
\declaretheorem[style=definition,qed=$\filledmedtriangleleft$, name=Aplicação, numberwithin=chapter]{ap}

\newtheorem*{nota}{Notação}

\theoremstyle{remark}
\newtheorem{remark}{Observação}[chapter]

\theoremstyle{plain}
\newtheorem{lema}{Lema}[chapter]
\newtheorem{prop}{Proposição}[chapter]
\newtheorem{cor}{Corolário}[chapter]


%----------------- INÍCIO DO DOCUMENTO -----------------
\begin{document}

%-------
%---CAPA

\begin{titlepage}
\begin{center}
\begin{figure}
    \centering
    \includegraphics[height=!, width= 2cm]{imagens/uepb.png}
\end{figure}

\textbf{UNIVERSIDADE ESTADUAL DA PARAÍBA\\
PRÓ-REITORIA DE PÓS-GRADUAÇÃO E PESQUISA \\
PROGRAMA DE PÓS-GRADUAÇÃO EM CIÊNCIA E TECNOLOGIA EM SAÚDE}

\vspace{4cm}


\textbf{RENAN MATIAS MOURA}   

\vspace{4cm}
\textbf{DESENVOLVIMENTO DE UMA API REST COM MODELO DE MACHINE LEARNING PARA PREDIÇÃO DE PRÉ-ECLÂMPSIA - PPML}

\vfill

\textbf{CAMPINA GRANDE\\
2025} 
    
\end{center}
\end{titlepage}

%----------------
%--FOLHA DE ROSTO

\thispagestyle{empty}
\begin{center}
 
    
\textbf{RENAN MATIAS MOURA}    

\vspace{5cm}

\textbf{DESENVOLVIMENTO DE UMA API REST COM MODELO DE MACHINE LEARNING PARA PREDIÇÃO DE PRÉ-ECLÂMPSIA - PPML }   

\vspace{5cm}

\begin{flushright}
\parbox{8cm}{
\singlespacing{\begin{hyphenrules}{nohyphenation}
Projeto encaminhado ao Comitê de Ética em Pesquisa da Universidade Estadual da Paraíba para análise e parecer com a finalidade de cumprir com investigações de estudo para para conclusão do mestrado em Ciência e tecnologia em saude.
\vspace{0.5cm}    

\end{hyphenrules}}}
\end{flushright}


\vspace{2cm}

\textbf{Orientador}: Prof. Dr. Frederico Moreira Bublitz.
\vfill


\textbf{CAMPINA GRANDE\\
2025}
\end{center}
\newpage

%----------------
%--FOLHA DE ROSTO (exemplo extra - mantenha/edite conforme necessário)
\thispagestyle{empty}
\begin{center}
\textbf{NOME DO ALUNO}

\vspace{1cm}

\textbf{TÍTULO DO TRABALHO}

\end{center}

\vspace{.4cm}
\begin{flushright}
\parbox{8cm}{
\singlespacing{\begin{hyphenrules}{nohyphenation}
Trabalho de Conclusão de Curso apresentado ao Departamento de XXXXXXXX do Centro de XXXXXXXX da Universidade Estadual da Paraíba como requisito parcial à obtenção do título de Licenciado(a) em XXXXXXX.
\vspace{0.5cm}

\textbf{Área de concentração}:  xxxxxxxxxx
\end{hyphenrules}}}
\end{flushright}

\vspace{.6cm}

\noindent Aprovado em: ---------

\vspace{1.5cm}

\begin{center}
\textbf{BANCA EXAMINADORA}
\vspace{1.5cm}

\noindent\rule{12cm}{0.4pt} \\
Prof. Dr. xxxxxxxxxxxxxx (Orientador)\\
Universidade Estadual da Paraíba (UEPB)
\vspace{2cm}

\noindent\rule{12cm}{0.4pt}\\
Profa. Dra. xxxxxxxxxxxxxx  \\
Universidade Estadual da Paraíba (UEPB)
\vspace{2cm}

\noindent\rule{12cm}{0.4pt}\\
Prof. Dr. xxxxxxxxxxxxxxx \\
Universidade XXXXXX (CASO MEMBRO EXTERNO)
\vspace{2cm}

\end{center}

%--------------
%---DEDICATÓRIA

\thispagestyle{empty}
\null
\vfill
\begin{flushright}
\parbox{4cm}{
\singlespacing{\begin{hyphenrules}{nohyphenation}
Dedicatória.
\end{hyphenrules}
}}
\end{flushright}
\thispagestyle{empty}
\newpage

%-----------------
%---AGRADECIMENTOS

\thispagestyle{empty}
\section*{\normalsize \centering \textbf{AGRADECIMENTOS}}

Escreva aqui os seus agradecimentos...

\newpage


%---------
%---RESUMO

\thispagestyle{empty}
\section*{\normalsize \centering \textbf{RESUMO}}

\noindent A pré-eclâmpsia (PE) é uma das causas mais comuns de morbidade e mortalidade materna e perinatal. A doença é caracterizada pelo desenvolvimento de hipertensão após 20 semanas de gestação e pode desencadear uma série de complicações. Em casos mais graves, a condição pode levar à morte, restrição de crescimento fetal e ao parto prematuro, aumentando significativamente o risco de complicações neonatais. Portanto, a detecção precoce e o manejo adequado da doença são essenciais para minimizar desfechos adversos tanto para a mãe quanto para o bebê. A Fetal Medicine Foundation (FMF) desenvolveu e disponibilizou a Calculadora de Risco para a PE, que combina fatores maternos, pressão arterial média e índices de pulsatilidade das artérias uterinas, além de marcadores séricos como o fator de crescimento placentário (PLGF), sendo amplamente reconhecida e utilizada desde o início dos anos 2010, sendo utilizados modelo de regressão logística. Nos últimos anos, técnicas de aprendizado de máquina têm emergido como ferramentas promissoras em aplicações médicas. A aplicação desses modelos baseia-se na análise de grandes volumes de dados clínicos, permitindo a identificação de padrões que podem indicar a doença antes mesmo do aparecimento de sintomas evidentes. Apesar dos avanços, as calculadoras foram validadas utilizando dados que podem não refletir a realidade brasileira, trazendo imprecisão ao diagnóstico, o que pode refletir na qualidade da assistência pré-natal e nos resultados perinatais. Visando esse problema, uma API com modelo de Machine Learning (ML) foi desenvolvida para avaliar a acurácia utilizando dados da nossa população. A ML foi treinada com dados clínicos reais, demonstrando resultados com alta precisão rápidas, podendo otimizar o acompanhamento às gestantes.

\vspace*{0.5cm}

\noindent\textbf{Palavras-chave:} Aprendizado de máquina. pré-eclâmpsia. doppler das artérias uterinas. triagem do primeiro trimestre.
 % no máximo 4 palavras-chave separadas entre si por ponto
 \newpage

%---------
%---ABSTRACT

\thispagestyle{empty}
\section*{\normalsize \centering \textbf{ABSTRACT}}

\noindent Resumo em inglês

\vspace*{0.5cm}

\noindent\textbf{Keywords:} Keyword 1. Keyword 2. Keyword 3.
\newpage

%----------
%---SUMÁRIO

\tableofcontents \thispagestyle{empty} 

%----------
%---INTRODUÇÃO

\chapter{INTRODUÇÃO}
\thispagestyle{myheadings}

Atualmente, com a inversão da pirâmide da assistência pré-natal, o foco passou a ser na detecção de pacientes com risco aumentado de desenvolver complicações durante a gestação e oferecer profilaxia, ao invés de esperar a complicação ocorrer e ser tratada. A detecção precoce de riscos gestacionais é fundamental para a saúde da mulher e da criança, principalmente em países como o Brasil, onde a morbimortalidade fetal, neonatal e materna ainda é significativa. Estudos indicam que, embora a cobertura da assistência pré-natal no Brasil seja praticamente universal, a adequação dessa teste assistência permanece baixa, com menos de 10\% das gestantes recebendo todos os procedimentos recomendados\cite{Viellas2014}.

Os distúrbios hipertensivos na gravidez estão entre as complicações mais comuns durante a gestação, englobando hipertensão crônica e hipertensão gestacional, sendo a pré-eclâmpsia (PE) a forma mais grave\cite{garovic2022}. Definida como um distúrbio multifatorial da gravidez, é caracterizada por disfunção placentária e dano vascular sistêmico, que coloca em risco a vida de mães e bebês em todo o mundo. Afetando cerca de 3\% a 5\% das gestações, chega a ser responsável por mais de 42 mil mortes maternas anuais, com impacto desproporcional em países de baixa e média renda, em que o acesso a cuidados obstétricos de qualidade são na sua maioria limitados\cite{chappell2021}.

A \textit{Fetal Medicine Foundation} (FMF) foi uma percursora nesta área e desenvolveu uma calculadora de risco para pré-eclâmpsia que combina dados da história materna, pressão arterial média e índices de pulsatilidade das artérias uterinas, além de outros biomarcadores, para estimar o risco de desenvolver pré-eclâmpsia durante a gestação. A calculadora utiliza o teorema de Bayes para combinar o risco prévio derivado de fatores maternos com dados adicionais coletados durante a gestação, permitindo uma avaliação mais precisa do risco de pré-eclâmpsia\cite{Riishede2023}. A utilização dessa ferramenta pode ajudar na identificação precoce de gestantes em risco, permitindo intervenções preventivas e melhorando os resultados maternos e fetais.

\begin{figure}[H]
    \centering
    \includegraphics[width=\textwidth, height=0.6\textheight, keepaspectratio]{imagens/FMF.png}
    \caption{Fonte: Fetal Medicine Foundation, 2025.}
\end{figure}


Pesquisas demonstram que modelos de aprendizado de máquina aplicados a dados de gestantes são capazes de classificar gravidez de risco com alta precisão, permitindo intervenções precoces\cite{silva2021avaliacao}, portanto identificar precocemente essas condições permite que os profissionais de saúde adotem medidas preventivas, ofereçam orientação adequada e garantam intervenções oportunas, reduzindo assim complicações durante a gestação e o parto. 

A importância de um sistema de avaliação de riscos gestacionais que possa ser utilizado no sistema público tem o potencial de salvar vidas, melhorar a qualidade dos cuidados maternos e promover a saúde das gestantes e de seus bebês em curto, médio e longo prazo. Portanto, a busca por soluções utilizando a Inteligência Artificial e modelos de avaliação de riscos gestacionais adaptados ao contexto brasileiro é um passo crucial na melhoria da saúde materna e perinatal no país. Iniciativas como a aplicação de modelos preditivos baseados em inteligência artificial têm mostrado eficácia na otimização dos cuidados neonatais e na redução da mortalidade materna\cite{alcino2024}.

Abordagens com ML para predição de PE (todos os tipos) têm demonstrado alta precisão, alcançando taxas de detecção acima dos 80\% e taxa de falsos positivos a 10\% superando as limitações de métodos anteriores que não consideravam variações populacionais. A capacidade de processar dados de forma dinâmica e fornecer recomendações baseadas em evidências em tempo real tem o potencial de revolucionar a aplicação clínica desses preditores, permitindo intervenções precoces e personalizadas. No entanto, para que esses modelos sejam amplamente adotados, é necessário superar desafios como a validação em diferentes populações e a integração de biomarcadores como o fator de crescimento placentário (PLGF) em contextos de atenção básica, onde o acesso a esses recursos ainda é limitado\cite{Gil2024}.  

Apesar da alta sensibilidade e especificidade apresentada pelos algoritmos disponibilizados pela FMF para prever PE, a validação externa é questionada devido ao fato de terem seus parâmetros calibrados a partir de dados europeus. Apesar de já ter sido validada para gestantes brasileiras, questiona-se se não poderíamos ter uma acurácia maior se utilizássemos dados da nossa população na calibração de seu algoritmo\cite{bilda2024}.

Portanto desenvolvemos um API treinada e testada, pronta para integração com outros softwares, utilizando um conjunto de informações do pré-natal e consultas ultrassonográficas, sem o uso de biomarcadores. Esses dados são originados do atendimento diario, no qual são armazenadas as informações dos pacientes em um sistema de laudos de ultrassom pré-existente a esse trabalho e um levantamento dos desfechos das gestações que foram acompanhadas na clínica privada EMBRION com sede em Campina Grande-PB.

%----------
%---CAPÍTULO I

\chapter{OBJETIVOS}
\thispagestyle{myheadings}

%----------

\section{Objetivos gerais}

Melhorar o acesso a ferramentas de predição de pré-eclâmpsia, especialmente em contextos com acesso limitado a biomarcadores, por meio de um sistema integrado a Inteligência Artificial (IA).

\section{Objetivos especificos}

Desenvolver uma API REST com um modelo de \textit{Machine Learning} (ML) treinado com dados maternos e fetais, sem uso de dados de biomarcadores, que seja capaz de predizer a pré-eclâmpsia com uma precisão melhor ou igual ao método da FMF. Como também oferecer uma integração REST com softwares de estabelecimentos de saúde que ofertam o serviço de pré-natal, visando a identificação precoce de gestantes com maior risco para desenvolvimento da PE. 

\chapter{RISCOS E BENEFICIOS}
\thispagestyle{myheadings}
\section{Benefícios}

O estudo tem como benefício principal a possibilidade de desenvolver um modelo preditivo de pré-eclâmpsia utilizando dados nacionais, contribuindo para a identificação precoce de gestantes em risco e auxiliando na melhoria dos desfechos materno-fetais, como também trará benefícios acadêmicos e científicos, com publicações em revistas nacionais e internacionais.

\section{Riscos}
Não existem riscos clínicos ou físicos aos participantes, uma vez que não haverá intervenção direta, apenas o uso de dados já existentes em prontuários eletrônicos. O risco identificado é de natureza informacional, referente à coleta, guarda e tratamento de dados clínicos em meio digital, especialmente considerando a utilização de armazenamento em dispositivos e em nuvem.

\section{Prevenção e minimização dos riscos}

\begin{itemize}
    \item Os dados utilizados não conterão informações pessoais diretas (como nome, CPF ou endereço), apenas variáveis clínicas armazenadas em momento anterior a pesquisa. 
    
    \item Cada participante será identificado apenas por um ID gerado no banco de dados no momento do cadastro do paciente na primeira consulta, garantindo o anonimato. 
    
    \item Durante a coleta e análise, os dados estarão disponíveis em nuvem com criptografia ativada e acesso controlado por IP, restrito apenas ao pesquisador responsável.
    
    \item Após a conclusão da coleta, os arquivos serão transferidos para um dispositivo local protegido por senha e criptografia, e o acesso externo será bloqueado novamente pela clínica responsável. 
    
    \item Os dados serão mantidos pelo período de 5 (cinco) anos após a finalização do estudo, conforme boas práticas científicas, sendo em seguida excluídos de forma definitiva.
\end{itemize}

Assim, os riscos relacionados ao uso em meio virtual são considerados mínimos e estão devidamente controlados, preservando a confidencialidade e garantindo a integridade das informações. 

%----------
%---fund. teorica

\chapter{FUNDAMENTAÇÃO TEÓRICA}
\thispagestyle{myheadings}

Para contextualizar e fundamentar o trabalho desenvolvido, esse capitulo aborda o conhecimento teórico sobre os teorias que fundamentam a abordagem do tema central do estudo."

\section{Pré-Natal}
O acompanhamento pré-natal configura-se como um pilar essencial para a redução da morbimortalidade materna e neonatal, integrando ações preventivas, diagnósticas e educativas. Estudos demonstram que consultas regulares diminuem em até 98\% os riscos de óbitos por causas evitáveis, como hemorragias e infecções puerperais, mediante a identificação precoce de condições como pré-eclâmpsia e diabetes gestacional\cite{Silva_2025}.

A atuação multidisciplinar no pré-natal revela-se crítica, com destaque para enfermeiros na detecção de riscos em Unidades Básicas de Saúde (UBS), onde consultas de enfermagem contribuem significativamente para reduzir complicações como partos prematuros, mediante monitoramento de pressão arterial, glicemia e orientações sobre aleitamento materno (SOUSA et al., 2024). Em casos de diabetes gestacional, o acompanhamento especializado permite ajustes dietéticos e farmacológicos que diminuem intervenções cirúrgicas emergenciais\cite{oliveira2025}.

Apesar dos avanços, desafios persistem na universalização do pré-natal qualificado. Pesquisas apontam que apenas 35\% das gestantes em regiões periféricas completam as seis consultas mínimas recomendadas, muitas vezes devido a barreiras geográficas e culturais\cite{Querioz2024}. Contudo, a carência de protocolos padronizados para abordagens específicas, como saúde bucal ou manejo de comorbidades, ainda limita a efetividade em escala nacional. Investimentos em capacitação profissional e tecnologias de monitoramento remoto surgem como alternativas promissoras, capazes de ampliar o acesso sem comprometer a qualidade do vínculo terapêutico\cite{Lima2024}.

\section{Pré-eclâmpsia}

A Pré-eclâmpsia (PE) é uma doença multissistêmica e uma das principais causas de morbidade e mortalidade materna e neonatal, contribuindo para cerca de 60.000 mortes maternas e mais de 500.000 nascimentos prematuros em todo o mundo a cada ano, sendo a condição marcada por hipertensão, podendo levar a várias outras complicações, incluindo convulsões (eclâmpsia), lesão renal aguda, disfunção hepática e até morte\cite{Silva2024Impactos}. Também pode resultar em desfechos neonatais adversos, como restrição do crescimento fetal e descolamento prematuro da placenta. Certos fatores aumentam o risco da gestante de desenvolver, como histórico de pré-eclâmpsia em gestações anteriores, obesidade, diabetes, hipertensão e gestações múltiplas, em que chega a atingir taxas variando de 8-20\% para gêmeos e 12-34\% para trigêmeos. Além disso, a condição é mais prevalente em populações com menor desenvolvimento, onde síndromes hipertensivas e complicações relacionadas são mais comuns \cite{Maayeh_2020}.

Estudos demonstram que a administração de aspirina na dose de 150 mg por dia, de 11 a 14 semanas até 36 semanas da gestação, resultou em uma notável redução de 62\% na incidência de PE prematura em gestantes com risco aumentado de desenvolver PE\cite{Rolnik_2017}. Como também a suplementação com cálcio demonstra eficácia na diminuição do risco de PE em 49\% e hipertensão gestacional em 30\%\cite{Jaiswal2024}. Tais achados reforçam a necessidade de modelos preditivos que selecionem grávidas com risco aumentado de desenvolver PE ainda no início da gestação, momento em que a profilaxia se mostrou eficaz.

\section{Biomarcadores}

Os biomarcadores clínicos são ferramentas que ampliam a acurácia diagnóstica e fundamentam a medicina de precisão, como observado no câncer de pulmão, no qual assinaturas genéticas orientam terapias direcionadas\cite{Xavier2022}.

Auxiliam no diagnóstico e monitoramento de doenças, permitindo que os médicos identifiquem e tratem condições de saúde de forma mais precisa e personalizada. No entanto, no Brasil, esses testes possuem um valor elevado, o que limita o acesso a muitos pacientes. Isso ocorre devido a uma combinação de fatores, incluindo a necessidade de importação, a complexidade tecnológica envolvida na sua detecção e análise, e os custos associados à regulamentação. Além disso, a infraestrutura de saúde no país pode não estar sempre equipada para realizar esses testes de forma eficiente, o que aumenta os custos operacionais e reflete no preço final para o paciente\cite{fonseca2023}%verificar ref.

\section{Padrões arquitetônicos de softwares }

\section{API REST}

Uma API REST (Representational State Transfer) é um padrão arquitetônico para comunicação entre sistemas que utiliza os princípios do protocolo HTTP para facilitar a comunicação entre clientes e servidores. Utiliza um conceito de “endpoints”, que são os pontos de comunicação do sistema, no qual se pode enviar ou receber dados atraves de uma URL. Segundo \cite{gowda2024}, sua eficiência reside na adoção de práticas como: (1) design orientado a recursos, em que entidades (como usuários ou produtos) são acessadas via  endpoints (ex: /api/v1/usuarios); (2) operações sem estado, garantindo que cada requisição contenha informações completas para processamento, eliminando dependências de sessões anteriores; e (3) uso explícito de métodos HTTP (GET, POST, PUT, DELETE) para mapear operações CRUD (Create, Read, Update, Delete). Essa abordagem traz benefícios significativos: escalabilidade, pois a ausência de conexões persistentes permite distribuir carga entre servidores; segurança robusta, com implementação de HTTPS e protocolos como JWT para autenticação; e flexibilidade, viabilizando integração facilitada entre sistemas. Além disso, técnicas como caching e paginação otimizam desempenho, reduzindo latência em cenários de alto tráfego\cite{gowda2024}. A adoção dessas práticas não apenas padroniza o desenvolvimento, mas também reduz riscos de vulnerabilidades e custos de manutenção, posicionando APIs REST como pilares essenciais para aplicações web modernas.

A integração de modelos preditivos por meio de APIs tem sido amplamente adotada para fornecer suporte à decisão médica em tempo real, facilitando a incorporação de inteligência artificial em sistemas hospitalares. APIs médicas permitem que dados clínicos sejam processados e analisados dinamicamente, proporcionando recomendações baseadas em evidências e aumentando a eficiência do diagnóstico\cite{mandel2016}. Além disso, a interoperabilidade entre sistemas eletrônicos de saúde possibilita que tais interoperabilidades sejam utilizadas em diversas plataformas, como prontuários eletrônicos e aplicativos móveis voltados para gestantes e profissionais de saúde\cite{himss2020}. Portanto, unindo a predição e a rápida integração facilitada com softwares já existentes, ou que podem vir a ser construídos, podemos tornar mais simples e efetivo esse acompanhamento clínico de gestantes em alto risco.

\section{Machine Learning}

 Machine Learning (ML) é uma subárea da Inteligência artificial (IA) que utiliza algoritmos computacionais para identificar padrões em dados e realizar predições, combinando métodos estatísticos e técnicas de ciência da computação. Na medicina, essa abordagem se divide em duas categorias principais: o aprendizado supervisionado, que prevê resultados (como o risco de infarto usando modelos como o Framingham Risk Score), e o aprendizado não supervisionado, que busca padrões ocultos em dados não rotulados (como a identificação de subtipos de doenças para medicina personalizada). Apesar do potencial, desafios significativos limitam sua aplicação prática: a necessidade de grandes conjuntos de dados diversificados, a dificuldade em selecionar variáveis biologicamente relevantes e a validação rigorosa dos modelos para garantir precisão em diferentes populações\cite{Deo2015}.

 Existem diversos modelos de ML, como: Random Forest, SVM (Support Vector Machine), Redes Neurais, entre outros. São algoritmos que aprendem com dados de entrada para fazer previsões ou decisões. Possui uma ampla gama de aplicações em vários campos, incluindo saúde, finanças, marketing e muito mais. Eles podem melhorar os produtos, otimizar processos e aprimorar as capacidades de tomada de decisão nas tarefas diárias.\cite{Sarker2021Ml}.

 O uso de ML tem se destacado na área da saúde como uma alternativa promissora para a predição de doenças, oferecendo maior precisão na análise de grandes volumes de dados clínicos\cite{Shickel2018}. Modelos como redes neurais, árvores de decisão e regressão logística têm sido aplicados na predição de complicações gestacionais, demonstrando capacidade de identificar padrões complexos que não seriam facilmente detectáveis por métodos convencionais\cite{Artzi2020}.  Estudos recentes indicam que modelos baseados em aprendizado supervisionado conseguem atingir altas taxas de sensibilidade e especificidade, quando treinados com bases de dados clínicos robustas e diversificadas\cite{Maric2020}.

%----------
%---CAPÍTULO III

\chapter{ESTADO DA ARTE}
\thispagestyle{myheadings}

Para aprofundar sobre o assunto, uma revisão de escopo esta sendo conduzida em paralelo a este trabalho com o intuito de aprofundarmos sobre a eficacia da aplicação da ML para o auxilo no diagnóstico precoce da PE. As buscas sobre os trabalhos co-relatos ao tema ocorreram em três bases online, PubMed, Embase (Elsevier) e Biblioteca virtual de saúde (BVS). O protocolo da revisão ja foi produzido e publicado, indicando as variaveis de buscas e mais detalhes (Anexo A).%adicionar a revisão de escopo no anexo A

No entanto, antes, precisamos abordar um pouco os métodos tradicionais de predição para a PE, que incluem a análise de fatores de risco clínicos, como idade materna, índice de massa corporal (IMC), antecedentes obstétricos e medidas da pressão arterial\cite{ACOG2020}. Além disso, exames laboratoriais e ultrassonográficos, como a dosagem do fator de crescimento placentário (PLGF) e a medida da artéria uterina por Doppler, têm sido utilizados para estimar o risco de desenvolvimento da condição\cite{Poon2014}. Essas abordagens apresentam limitações, como alto custo e baixa acessibilidade, dificultando sua ampla adoção em países de baixa e média renda\cite{who2019}.

As calculadoras de riscos também são utilizadas para prever a PE e empregam métodos de regressão integrando um conjunto de variáveis preditivas, incluindo características maternas, pressão arterial média (PAM), índice de pulsatilidade da artéria uterina (UTA-PI), PLGF e proteína plasmática A associada à gravidez (PAPP-A) para estimar riscos obstétricos. Desenvolvidas em colaboração entre a Fetal Medicine Foundation (FMF) e o Hospital Clínic de Barcelona, essas ferramentas demonstraram maior precisão diagnóstica em comparação com métodos tradicionais, além de serem a única abordagem com validação internacional\cite{Riishede2023}.

O estudo de \cite{Rezende2024}, realizado no brasil, mostra que mesmo validados, métodos como esse não levam em conta a população local para calibrar seus parâmetros fundamentais, o que é essencial para uma boa precisão em tipos de populações distintas, no qual o resultado é uma taxa considerada alta de falsos positivos (FPR) de 24,4\%. Quando comparamos com outros modelos preditivos que utilizam ML, que obtêm resultados de FPR abaixo de 15\%\cite{Gil2024}. Com isso podemos observar a discrepância desses resultados, demonstrando que a falta de dados personalizados para a população afeta expressivamente o nível de precisão do resultado. Apesar disso, o modelo da FMF mantém relevância clínica por sua validação internacional e simplicidade operacional, sendo amplamente adotado em contextos de recursos limitados e frequentemente referenciadas por outros autores que desenvolvem modelos de aprendizado de máquina como um padrão para validação de sua acurácia, por ser um padrão aceito e validado\cite{Torres-Torres}.

Além disso, Araújo et al. propuseram um modelo de aprendizado de máquina baseado em LightGBM, treinado com dados sintéticos gerados pela metodologia DAS (Data Augmentation and Smoothing) a partir de parâmetros do hemograma completo (CBC), para apoiar o diagnóstico de pré-eclâmpsia com sinais de gravidade\cite{Araujo2024}. O modelo alcançou AUROC de 0,90, sensibilidade de 0,95 e especificidade de 0,79, evidenciando o alto potencial preditivo dessa combinação. No entanto, por depender de exame laboratorial, ainda adiciona uma etapa e um requisito de infraestrutura ao fluxo assistencial, o que pode limitar sua aplicabilidade imediata em cenários com recursos mais restritos.

Demonstrando o potencial das tecnologias de IA na transformação do cuidado pré-natal um estudo desenvolveu usando Python, scikit-learn e TensorFlow, com dados analisados por meio do SPSS 22 uma versão do IBM SPSS Statistics para prever o risco de PE. Enquanto algoritmos de deep learning e Extra Trees Classifier, foram empregados para avaliar o poder preditivo de diferentes variáveis. O estudo utilizou técnicas como aumento de gradiente e otimização de hiper parâmetros para melhorar o desempenho do modelo, que demonstrou uma sensibilidade de 73,7\% e especificidade de 92,7\%, indicando sua eficácia na distinção entre casos positivos e negativos para pré-eclâmpsia\cite{Bulez2024}.

Recente estudo chinês testou cinco algoritmos de classificação diferentes, Logistic Regression, Extra Trees Classifier, Voting Classifier, Gaussian Process Classifier e Stacking Classifier. O Stacking Classifier apresentou o melhor desempenho na previsão da pré-eclâmpsia prematura, com uma área sob a curva (AUC) de 0,884. no qual incorpora várias características maternas, como idade, altura, peso antes da gravidez e biomarcadores clínicos, como PAM, UTA-PI, PAPP-A e PLGF, utilizando da ferramenta SHAP (ShaPley Additive Explanations) para explicar as previsões do modelo e oferecer mais transparência do modelo, o que torna mais seguro a aplicação em contextos clínicos, porém ressalta a necessidade de validação desse método por considerar dados de apenas um único centro local\cite{Li2024}.

Portanto, fica evidente a viabilidade e relevância do desenvolvimento de uma nova ferramenta preditiva para avaliação do risco de pré-eclâmpsia baseada em algoritmos de machine learning, utilizando dados clínicos específicos da nossa população. Essa abordagem personalizada tem o potencial de superar as limitações dos modelos atuais, especialmente no que se refere à taxa de falsos positivos e à acurácia em contextos locais, contribuindo significativamente para a tomada de decisão clínica mais precisa e eficaz e facilitar a implementação dessas ferramentas em diferentes realidades socioeconômicas. Diante deste cenário, construimos um modelo de ML, o Lightgbm, implementado em uma API REST para prever a PE, elegemos esse modelo após uma revisão sistematica da eficacia do machine learning na predição da PE no geral, e para a nossa realidade do tamanho da amostra que iremos trabalhar os algoritmos da familia do gradient boosting(GBM) se mostraram sempre acertivos e sem complexidade para o treinamento.

\setlength{\tabcolsep}{4pt} % diminui espaçamento lateral
\renewcommand{\arraystretch}{1.25} % melhora leitura vertical

\begin{table}[H]
\centering
\caption{Comparação entre modelos de predição de pré-eclâmpsia descritos na literatura e o presente estudo.}
\small

\begin{tabularx}{\textwidth}{
 >{\RaggedRight\arraybackslash}p{2.6cm}
 >{\RaggedRight\arraybackslash}X
 >{\centering\arraybackslash}p{1.4cm}
 >{\centering\arraybackslash}p{1.9cm}
 >{\centering\arraybackslash}p{2.4cm}
 >{\RaggedRight\arraybackslash}X
}

\hline
\textbf{Estudo / Modelo} &
\textbf{Variáveis} &
\textbf{AUC} &
\textbf{Sensibilidade} &
\textbf{Especificidade / FPR} &
\textbf{Observações} \\
\hline

FMF (regressão) \cite{Riishede2023,Rezende2024}
& Dados maternos; PAM; UTA-PI; PLGF; PAPP-A
& --
& --
& FPR = 24,4\%
& Modelo validado internacionalmente; elevada taxa de falsos positivos na população brasileira. \\

Araújo et al. (LightGBM) \cite{Araujo2024}
& Hemograma completo (CBC) com dados sintéticos (DAS)
& 0,90
& 0,95
& 0,79 (esp.)
& Requer exames laboratoriais; maior complexidade operacional. \\

Bulez et al. (DL / Extra Trees) \cite{Bulez2024}
& Variáveis clínicas diversas
& --
& 73,7\%
& 92,7\% (esp.)
& Deep Learning com otimização extensa de hiperparâmetros. \\

Li et al. (Stacking) \cite{Li2024}
& Dados maternos; PAM; UTA-PI; PLGF; PAPP-A
& 0,884
& --
& --
& Validação em centro único; uso de SHAP para explicabilidade. \\

\textbf{Presente estudo (LightGBM)}
& \textbf{25 variáveis clínicas/obstétricas (peso, IMC, Idade, diabetes, HAS, Doppler, comorbidades)}
& \textbf{0,9669}
& \textbf{0,91}
& \textbf{0,955 (esp.) / FPR = 4,5\%}
& \textbf{Amostra real local; API clínica leve e sem necessidade de exames laboratoriais.} \\

\hline
\end{tabularx}

\end{table}


%--------------
%---CAPÍTULO VI

\chapter{PPML – API REST FULL}
\thispagestyle{myheadings}

Foi denominada PPML e desenvolvida em \textit{Python} 3.9+ utilizando \textit{FastAPI} como \textit{framework web} assíncrono, com validação automática de entrada via Pydantic e documentação OpenAPI/Swagger nativa. A comunicação ocorre via HTTP/HTTPS com payloads JSON, suportando deploy containerizado utilizando Uvicorn, e publicado no serviço de nuvem do \textit{Heroku}

O modelo final implementado foi o LightGBM(LGBMClassifier) de Gradiente Boosting otimizado para classificação binária, por demonstrar resultados com mais precisão com o tamanho da amostra disponivel, treinado com 25 features clínicas/obstétricas(peso, idade, IMC, diabetes, hipertensão, percentis Doppler, comorbidades). Configurando com n\_estimators=500, learning\_rate=0.1, num\_leaves=31, max\_depth=6 e regularização (reg\_alpha/lambda=0.1), alcançou AUC-ROC 0.9669, precisão/recall/F1 0.9091 no conjunto de teste independente(22+22 casos), com early stopping em 800 iterações.

Também foi desenvolvido e avaliado um modelo baseado em Rede Neural para classificação binária, treinado com as mesmas 25 variáveis clínicas e obstétricas, utilizando normalização dos dados e balanceamento por oversampling. Entretanto, com o tamanho limitado da amostra disponível, o modelo apresentou capacidade discriminatória inferior à do método de Gradiente Boosting. A avaliação no conjunto de teste mostrou AUC-ROC = 0.6548, indicando baixo poder de separação entre os desfechos. A análise por diferentes limiares de decisão evidenciou desempenho instável, com destaque para o melhor ponto em threshold = 0.20, no qual foram obtidos precision = 0.292, recall = 1.000 e F1-score = 0.452. Em limiares mais conservadores, observou-se queda acentuada da sensibilidade, como em threshold = 0.40, com precision = 0.250, recall = 0.429 e F1-score = 0.316, e em threshold = 0.50, com precision = 0.500, recall = 0.286 e F1-score = 0.364. Esses resultados demonstram que, apesar da alta sensibilidade obtida em limiares baixos, a rede neural apresentou elevado número de falsos positivos e desempenho global insatisfatório para aplicação clínica isolada, reforçando a superioridade do modelo baseados em \textit{Gradient Boosting} para o contexto estudado.

O pré-processamento do modelo implementado segue pipeline customizada com pandas (manipulação), \textit{scikit-learn}, \textit{StandardScaler} (normalização), mapeamentos categóricos (raça: Branco=1/Pardo=2/Preto=3; histórico diabetes), extração de comorbidades via \textit{regex} de acordo com as entradas categorizadas, preenchimento NaN (0/moda), e garantia de 25 features fixas.

O treinamento incorporou SMOTE (balanceamento 15.9\%→50\% positivos), GroupShuffleSplit por paciente para evitar data leakage, data augmentation gaussiana, onde aumentamos +40\% nas features contínuas e splits estratificados, separando para treino um total de 72\%, validação 13\%, teste 15\%. Avaliação incluiu ROC-AUC, matriz confusão, curvas aprendizado e SHAP-like feature importance, onde no top 3: peso, diabetes, idade).

Endpoints expõem predict\_lgbm (probabilidade + classificação risco),health (status modelos) e features (esquema entrada), com threshold 0.5 para alto/baixo risco e tratamento robusto de exceções. disponibilizaremos o serviço através de um servidor Linux de alta disponibilidade, hospedado no Heroku, uma plataforma em nuvem de baixo custo, podendo também ser utilizado como ambiente de treinamento e teste dos modelos.

\begin{quadro}[H]
\centering

{\bfseries Quadro 1: Dados de entrada para predição (continua)}

\vspace{2mm}

\begin{tabular}{|l|}
\hline
Peso (kg) \\ \hline
Idade \\ \hline
Índice de Massa Corporal (IMC) \\ \hline
Diabetes \\ \hline
Hipertensão \\ \hline
Peso no primeiro trimestre \\ \hline
Origem racial \\ \hline
Histórico familiar de diabetes \\ \hline
Índice de pulsatilidade médio da artéria uterina (IP médio) \\ \hline
História obstétrica anterior \\ \hline
Perdas gestacionais \\ \hline
Tabagismo durante a gravidez \\ \hline
Idade gestacional \\ \hline
Idade gestacional corrigida \\ \hline
Peso fetal \\ \hline
Percentil da artéria uterina \\ \hline
Percentil do peso fetal \\ \hline
\end{tabular}

\vspace{2mm}
\footnotesize
Fonte: Adaptado de Fetal Medicine Foundation, 2025.
\end{quadro}

\begin{quadro}[H]
\centering

{\bfseries Quadro 1: Dados de entrada para predição (continuação)}

\vspace{2mm}

\begin{tabular}{|l|}
\hline
Hipertensão pré-existente \\ \hline
Endometriose \\ \hline
Hipotireoidismo \\ \hline
Doença inflamatória intestinal (IIC) \\ \hline
Trombofilia \\ \hline
Retocolite \\ \hline
Gastrite \\ \hline
Outras doenças associadas \\ \hline
\end{tabular}

\vspace{2mm}
\footnotesize
Fonte: Adaptado de Fetal Medicine Foundation, 2025.
\end{quadro}


O projeto foi organizado em pastas funcionais básicas para facilitar manutenção e desenvolvimento ágil, no qual a pasta API contém os endpoints RESTful, gerenciando recebimento e validação dos dados e respostas. já a pasta DATA armazena a planilha utilizada para o treinamento e para alimentar o modelo. A pasta MODELS armazena o modelo treinado, o LightGBM serializado, e scripts para carregamento e inferência. a pasta NOTEBOOKS é usada para experimentação, treinamento, avaliação e geração dos modelos, incluindo análise exploratória e métricas. A organização do codigo foi pensada para seguir uma organização simples, seguindo os principios do Clean code, garantindo separação das responsabilidades e modularidade.

\begin{figure}[H]
    \centering
    \includegraphics[width=0.7\textwidth]{imagens/pastas.png}
    \caption{Fonte: Autor, 2025.}
\end{figure}


\chapter{METODOLOGIA}
\thispagestyle{myheadings}


A metodologia do desenvolvimento da PPML seguirá o processo de pipeline de Machine Learning, baseado no modelo proposto por \cite{Kreuzberger2023MLOps}, adaptado à nossa proposta de aplicação clínica.
\begin{figure}[H]
    \centering
    \includegraphics[width=\textwidth, height=0.6\textheight, keepaspectratio]{imagens/pipeline.png}
\end{figure}

\section{Análise dos dados}

Os dados utilizados neste estudo são provenientes do banco de dados relacional da clínica privada EMBRION, composto por laudos ultrassonográficos e informações clínicas de gestantes atendidas na instituição. Cada registro é associado a um identificador único de paciente (\textit{PacienteId}), não havendo acesso a informações diretamente identificáveis, o que garante a confidencialidade das participantes. Inicialmente, foi realizada uma análise exploratória para quantificar o número de pacientes e exames disponíveis, caracterizar a distribuição do desfecho de pré-eclâmpsia e inspecionar a presença de valores ausentes, inconsistências em datas (por exemplo, datas fora do intervalo gestacional plausível) e possíveis erros de digitação em variáveis contínuas, como peso, índice de massa corporal (IMC) e medidas fetais.

Essa etapa permitiu definir quais tipos de exames ultrassonográficos seriam considerados (como obstétrico simples, obstétrico com Doppler e translucência nucal), bem como avaliar a viabilidade de aproveitar múltiplos exames por gestante ao longo da gestação. Também foram identificadas as principais variáveis clínicas de interesse, tais como idade materna, presença de diabetes e hipertensão, histórico obstétrico e tabagismo, de modo a orientar as etapas subsequentes de preparação dos dados e seleção de atributos.

\begin{figure}[H]
    \centering
    \includegraphics[width=\textwidth, height=0.6\textheight, keepaspectratio]{imagens/diagrama_embrion.png}
    \caption{Fonte: Fetal Medicine Foundation, 2025.}
\end{figure}

\section{Preparação dos dados e validação}

Após a etapa exploratória, os dados clínicos e ultrassonográficos foram exportados dos sistemas da clínica e unificados em uma única planilha, na qual as diferentes tabelas foram integradas por meio do campo \textit{PacienteId}. A preparação dos dados envolveu a conversão explícita dos tipos de variáveis (por exemplo, datas, números inteiros e reais), a padronização de categorias textuais (como cor/raça, histórico familiar de diabetes e presença de comorbidades) e o cálculo da idade materna a partir da data de nascimento e de uma data de referência consistente para cada paciente. Valores ausentes foram tratados com regras específicas: uso da moda para variáveis categóricas, substituição por zero para variáveis que representam ausência de condição clínica e utilização de valores de referência quando não foi possível inferir a informação a partir dos registros.

Além disso, foram aplicados limites fisiologicamente plausíveis às variáveis contínuas, com o objetivo de mitigar o impacto de erros de digitação sem descartar registros inteiros (por exemplo, restringindo a idade materna a um intervalo entre 15 e 50 anos, peso entre 40 e 150 kg, IMC entre 15 e 50, e peso fetal máximo de 5\,000 g). A partir de textos livres sobre antecedentes clínicos, foram criadas variáveis binárias indicadoras de condições específicas, como hipertensão pré-existente, trombofilia, endometriose, doenças tireoidianas e outras comorbidades relevantes. A versão final do conjunto de dados pré-processado foi submetida à validação de uma especialista em medicina fetal, que avaliou a coerência das decisões de codificação, inclusão e exclusão de variáveis, bem como dos limites adotados, à luz do conhecimento clínico atual sobre fatores de risco para pré-eclâmpsia.

\section{Treinamento do modelo}

Concluída a preparação dos dados, procedeu-se à definição da variável alvo (\textit{PreEclampsia}) e do conjunto de atributos de entrada, compostos por variáveis clínicas e ultrassonográficas selecionadas na etapa anterior. Para evitar vazamento de informação entre conjuntos de treino e teste, foi adotada uma estratégia de divisão baseada em grupos, garantindo que todos os registros de uma mesma gestante permanecessem em apenas um dos conjuntos. Em seguida, foram construídos os conjuntos de treino, validação e teste, com estratificação em relação ao desfecho para preservar a proporção de casos de pré-eclâmpsia em cada partição.

\setlength{\tabcolsep}{4pt}
\renewcommand{\arraystretch}{1.25}

\begin{table}[H]
\centering
\caption{Variáveis maternas e ultrassonográficas utilizadas no estudo.}
\small

\begin{tabularx}{\textwidth}{
  >{\RaggedRight\arraybackslash}p{3.2cm}
  >{\RaggedRight\arraybackslash}X
}

\hline
\textbf{Fonte} & \textbf{Variável} \\
\hline

Dados maternos/ultrassonográficos &
Paciente ID \\
& Data \\
& Peso \\
& IMC \\
& Diabetes \\
& Hipertensão \\
& Data de Nascimento \\
& Data Provável do Parto \\
& Peso Primeiro Trimestre \\
& Origem Racial \\
& Histórico Familiar de Diabetes \\
& Risco Trissomia 21 \\
& Risco Trissomia 18 \\
& Risco Basal Pré-Eclâmpsia \\
& Risco Corrigido Pré-Eclâmpsia \\
& Risco Restrição de Crescimento \\
& Percentil \\
& Média IP – Percentil Artéria uterina \\
& História Obstétrica Anterior \\
& Perdas Gestacionais \\
& Fuma \\
& Idade Gestacional 2 \\
& Idade Gestacional Corrigida 2 \\
& Peso Fetal \\
& Medicação \\
& Uso de Medicação \\
& Outro Uso de Medicação \\
& Maior Bolsão \\
& Circunferência Abdominal \\
& Percentil Circunferência Abdominal \\
& Percentil Artéria Umbilical \\
& Percentil Relação Cérebro–Placenta \\
& Percentil Artéria \\
& Percentil Peso \\
& Data Provável do Parto 1º Trimestre \\
& Doença Pré-Existente \\
& Tipo de Doença Pré-Existente \\
& Biometria Embrionária (dias) \\
& Biometria Embrionária (semanas) \\
\hline
\end{tabularx}

\end{table}

\setlength{\tabcolsep}{4pt}
\renewcommand{\arraystretch}{1.25}

\begin{table}[H]
\centering
\caption{Variáveis do questionário de desfechos utilizadas no estudo}
\small

\begin{tabularx}{\textwidth}{
  >{\RaggedRight\arraybackslash}p{3.2cm}
  >{\RaggedRight\arraybackslash}X
}
\hline
\textbf{Fonte} & \textbf{Variável} \\
\hline

Dados do questionário de desfechos &
Paciente ID \\
& Pré-eclâmpsia \\
& Diabetes \\
& Anemia \\
& Hipotireoidismo \\
& Outras Doenças \\
& Medicamentos \\
& Último Peso \\
& Idade Gestacional \\
& Tipo Parto \\
& Data Nascimento \\
& Peso Nascimento \\
& Comprimento ao nascer \\
& Perímetro Cefálico \\
& Apgar 1 minuto \\
& Apgar 5 minutos \\
& Dias Hospital \\
& Intercorrências \\
& Tipo Diabetes \\
& Primeiro Peso gravidez \\
& Data Parto \\
& Data Peso Cartão \\
\hline
\end{tabularx}

\end{table}



Devido ao desbalanceamento natural dos dados (menor frequência de casos de pré-eclâmpsia em relação aos controles), empregou-se a técnica de \textit{Synthetic Minority Over-sampling Technique} (SMOTE) para geração de amostras sintéticas da classe minoritária, seguida de uma etapa de \textit{data augmentation} numérica, na qual foi adicionado ruído gaussiano controlado às variáveis contínuas, respeitando os limites fisiológicos previamente definidos. O modelo de aprendizado de máquina escolhido foi o \textit{Light Gradient Boosting Machine} (LightGBM), configurado para classificação binária, utilizando \textit{early stopping} com base no desempenho no conjunto de validação. Os hiperparâmetros foram ajustados com foco em maximizar a sensibilidade (recall) para o desfecho de pré-eclâmpsia, mesmo ao custo de maior número de falsos positivos, em consonância com a priorização clínica de reduzir falhas na detecção de gestantes de alto risco.

\section{Validação do modelo}

A avaliação de desempenho do modelo foi conduzida de forma independente nos conjuntos de validação e teste, que não participaram do treinamento nem do balanceamento das classes. Para cada paciente, o modelo produziu probabilidades de ocorrência de pré-eclâmpsia, a partir das quais foram estimadas curvas \textit{Receiver Operating Characteristic} (ROC) e \textit{precision-recall}. A partir dessas curvas, foram calculadas métricas de desempenho, incluindo área sob a curva (AUC), acurácia, sensibilidade, especificidade, precisão, \textit{F1-score} e matriz de confusão, com ênfase na análise da sensibilidade em diferentes pontos de corte.

Com base nessas análises, foi definido um limiar de decisão para a probabilidade de pré-eclâmpsia que equilibra adequadamente sensibilidade e especificidade, priorizando a identificação de gestantes em maior risco. Esse limiar foi discutido com a especialista em medicina fetal, considerando o impacto clínico da classificação de risco, o fluxo de atendimento da clínica e as recomendações de diretrizes nacionais sobre triagem e prevenção de pré-eclâmpsia. Adicionalmente, foram analisadas as importâncias das variáveis fornecidas pelo LightGBM, permitindo identificar quais atributos contribuíram de forma mais relevante para a predição e apoiar a interpretação clínica do modelo.

\section{Desenvolvimento da API}

Após a definição do modelo final e do limiar de decisão, o classificador LightGBM, juntamente com o conjunto de pré-processamento (funções de limpeza, codificação e normalização) e a lista ordenada de atributos, foi empacotado em artefatos serializados para uso em produção. Com base nesses artefatos, foi desenvolvida uma interface de programação de aplicações (API) em linguagem Python, responsável por expor serviços de predição por meio de requisições HTTP. A API foi implementada de forma a receber, em formato \textit{JSON}, os dados clínicos e ultrassonográficos necessários, aplicar o mesmo fluxo de pré-processamento descrito na etapa de preparação dos dados e, em seguida, invocar o modelo treinado para obter a probabilidade estimada de pré-eclâmpsia para cada gestante.

A resposta da API inclui tanto a probabilidade contínua quanto a classificação binária de risco (alto ou baixo), baseada no limiar de decisão definido na etapa de validação. Opcionalmente, a API também pode fornecer informações agregadas sobre a importância relativa das variáveis, com o intuito de apoiar a interpretação do resultado pelo profissional de saúde. Toda a comunicação com o sistema de prontuário da clínica é realizada utilizando apenas o identificador interno do paciente, preservando a confidencialidade dos dados. Essa arquitetura permite integrar o modelo de predição ao fluxo de trabalho assistencial, oferecendo suporte à tomada de decisão em tempo quase real durante o atendimento.

\section{Comparação com diagnósticos reais}

Em ambiente de uso rotineiro, cada predição realizada pela API é armazenada em uma tabela específica, contendo o identificador da paciente, a data da predição, a probabilidade estimada de pré-eclâmpsia e a classificação de risco atribuída pelo modelo. Após o desfecho da gestação, os diagnósticos reais e os principais desfechos perinatais (como idade gestacional ao parto, peso ao nascimento, necessidade de internação e complicações associadas) são vinculados a essas predições, possibilitando a comparação sistemática entre as saídas do modelo e os resultados observados na prática clínica.

Essa comparação contínua permite monitorar o desempenho do modelo ao longo do tempo, detectar eventuais degradações de acurácia decorrentes de mudanças no perfil da população atendida ou nos protocolos assistenciais e, quando necessário, planejar ciclos de re-treinamento com a incorporação de novos dados. Dessa forma, estabelece-se um ciclo de melhoria contínua, no qual os dados gerados na rotina assistencial alimentam sucessivas versões do modelo, garantindo que o sistema de apoio à decisão se mantenha atualizado e alinhado às evidências e diretrizes mais recentes sobre predição e prevenção da pré-eclâmpsia.




\cleardoublepage
\phantomsection
\addcontentsline{toc}{chapter}{REFERÊNCIAS}
\begin{center}
\bfseries\MakeUppercase{Referências}
\end{center}
\vspace{1cm}
\printbibliography[heading=none]

%------------
%---APÊNDICES

\cleardoublepage
\addcontentsline{toc}{chapter}{APÊNDICES}
\appendix
\setcounter{chapter}{0}
\renewcommand{\thechapter}{\Alph{chapter}}

\chapter*{\hfill APÊNDICE A -- Protocolo da revisão de escopo \hfill}
\stepcounter{chapter}
\thispagestyle{myheadings}

% Insere todas as páginas do PDF
\includepdf[
  pages=-,
  pagecommand={}
]{imagens/effectiveness_ml_predicting.pdf}



\end{document}
